\documentclass[12pt,preprint]{aastex}
%\documentclass{emulateapj}
\usepackage{amssymb,amsmath}
%\usepackage{caption}
%\DeclareCaptionLabelSeparator{dotemdash}{.--- }
%\captionsetup[figure]{labelformat=simple, labelsep=dotemdash}
%\usepackage{color,hyperref}
% hypertex insanity
%\definecolor{linkcolor}{rgb}{0,0,0.25}
%\hypersetup{
%  colorlinks=true,        % false: boxed links; true: colored links
%  linkcolor=linkcolor,    % color of internal links
%  citecolor=linkcolor,    % color of links to bibliography
%  filecolor=linkcolor,    % color of file links
%  urlcolor=linkcolor      % color of external links
%}
\newcounter{address}
\setcounter{address}{1}
%\usepackage{sidecap}
\setlength{\emergencystretch}{2em}%No overflowing references
\newcommand{\ie}{i.e.}
\newcommand{\etal}{et al.}
\newcommand{\dd}{\mathrm{d}}
\newcommand{\eg}{e.g.}
\newcommand{\eqnname}{equation}
\newcommand{\equationname}{\eqnname}
\renewcommand{\tablename}{Table}
\renewcommand{\figurename}{Figure}
\newcommand{\figurenames}{\figurename~s}
\newcommand{\sectionname}{$\mathsection$}
\newcommand{\normal}{\ensuremath{\mathcal{N}}}
\newcommand{\flag}[1]{\texttt{\lowercase{#1}}}
\newcommand{\feh}{\ensuremath{[\mathrm{Fe/H}]}}
\newcommand{\afe}{\ensuremath{[\alpha\mathrm{/Fe}]}}
\newcommand{\logg}{log g}
\newcommand{\Ro}{\ensuremath{R_0}}

\renewcommand{\vec}[1]{\ensuremath{\mathbf{#1}}}
\newcommand{\vecx}{\ensuremath{\vec{x}}}
\newcommand{\vecv}{\ensuremath{\vec{v}}}
\newcommand{\vecj}{\ensuremath{\vec{J}}}
\newcommand{\veco}{\ensuremath{\vec{\Omega}}}
\newcommand{\veca}{\ensuremath{\vec{\theta}}}
\newcommand{\df}{\ensuremath{f}}
\newcommand{\tdf}{\ensuremath{\mathrm{tDF}}}
\newcommand{\paramsdf}{\ensuremath{\vec{p}_{\mathrm{DF}}}}
\newcommand{\paramspot}{\ensuremath{\vec{p}_\Phi}}

\newcommand{\DF}{\df}
\newcommand{\dff}{\ensuremath{\mathbf{f}}}
\newcommand{\dex}{\ensuremath{\,\mathrm{dex}}}
\newcommand{\Gyr}{\ensuremath{\,\mathrm{Gyr}}}
\newcommand{\kpc}{\ensuremath{\,\mathrm{kpc}}}
\newcommand{\pc}{\ensuremath{\,\mathrm{pc}}}
\newcommand{\kms}{\ensuremath{\,\mathrm{km\ s}^{-1}}}
\newcommand{\msun}{\ensuremath{\,\mathrm{M}_{\odot}}}
\newcommand{\inv}{\ensuremath{^{-1}}}

\newcommand{\jr}{\ensuremath{J_R}}
\newcommand{\jphi}{\ensuremath{J_\phi}}
\newcommand{\jz}{\ensuremath{J_Z}}
\newcommand{\lz}{\ensuremath{L_Z}}
\newcommand{\Or}{\ensuremath{\Omega_R}}
\newcommand{\Ophi}{\ensuremath{\Omega_\phi}}
\newcommand{\Oz}{\ensuremath{\Omega_Z}}
\newcommand{\ar}{\ensuremath{\theta_R}}
\newcommand{\aphi}{\ensuremath{\theta_\phi}}
\newcommand{\az}{\ensuremath{\theta_Z}}

%\submitted{}

\begin{document}

\title{Dynamical modeling of tidal streams}
\author{Jo~Bovy\altaffilmark{1,2,3} \etal}
%  Hans-Walter~Rix\altaffilmark{4}}
\altaffiltext{\theaddress}{\label{IAS}\stepcounter{address} Institute
  for Advanced Study, Einstein Drive, Princeton, NJ 08540, USA}
\altaffiltext{\theaddress}{\label{Hubble}\stepcounter{address} Hubble
  fellow}
\altaffiltext{\theaddress}{\label{email}\stepcounter{address}
  Correspondence should be addressed to bovy@ias.edu~.}
%\altaffiltext{\theaddress}{\label{MPIA}\stepcounter{address}
%  Max-Planck-Institut f\"ur Astronomie, K\"onigstuhl 17, D-69117
%  Heidelberg, Germany}

\begin{abstract} 
  We present a generative, dynamical model for a tidal stream.
\end{abstract}

\keywords{
	Galaxy: fundamental parameters
	---
        Galaxy: kinematics and dynamics
        ---
	Galaxy: structure
}


\section{Introduction}


%BOVY: reference point


\section{Two equivalent generative models of tidal streams}\label{sec:method}


%BOVY: reference point

\acknowledgements It is a pleasure to thank \dots for helpful comments
and assistance. J.B. was supported by NASA through Hubble Fellowship
grant HST-HF-51285.01 from the Space Telescope Science Institute,
which is operated by the Association of Universities for Research in
Astronomy, Incorporated, under NASA contract NAS5-26555. J.B. 
%and H.-W.R  (BOVY: ALSO ACKNOWLEDGE WITHOUT S)
acknowledges support from SFB 881 (A3) funded by the German
Research Foundation DFG.


\appendix

\section{An efficient, general method for calculating action-angle coordinates using orbit integration}\label{sec:aa}


\section{Dynamical modeling of two-star streams}\label{sec:twostar}

%BOVY: reference point

\begin{thebibliography}{}

\bibitem[{{Binney} \& {Tremaine}(2008)}]{binneytremaine}
  Binney,~J. \& Tremaine,~S. 2008, Galactic Dynamics: Second Edition
\end{thebibliography}


\end{document}
